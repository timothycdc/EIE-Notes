\chapter{Towards Learning in Linear Models}


\section{Supervised Learning}
Supervised learning settings assume that we have access to input-output pairs, $(\bm{x}, \bm{y})$, with $\bm{x} \in \mathbb{R}^n$ and $\bm{y} \in \mathbb{R}^m$. It is typically assumed that we have many such pairs comprising a dataset $\{(\bm{x}^{(i)}, \bm{y}^{(i)})\}_{i=1}^N$.

\section{Linear Regression Model}
Linear regression is exactly what the name suggests: performing regression with a linear model. Assuming a domain of $\mathbb{R}^n$ and a one-dimensional co-domain, we can write our model as $f(\bm{x}) = \bm{x}^\top \bm{\theta}$. Thus our model can be written as:

\[
\hat{y}^{(i)} = \bm{x}^{(i)\top} \bm{\theta}
\]

The key idea behind learning is to find $\bm{\theta}$ such that $\hat{y}^{(i)} \approx y^{(i)}$.You may recall linear models written as affine transformations: $y = mx + b$, where $b$ is the bias or constant term – this makes the model affine and not linear. Linear models refers to the relationship between model parameters and predictions via a linear transformation.

\marginnote[17.5pt]{
    \defsb{Affine Transformations}{
        Affine transformations are more general than linear transformations, because they include not only scaling and rotation, but also translations.
    }
}


\defb{Linear Transformation}{
A linear transformation between two vector spaces $V$ and $W$ is a map 

$$T : V \rightarrow W$$

such that:

\begin{itemize}
    \item $T(v_1 + v_2) = T(v_1) + T(v_2) \quad \forall v_1, v_2 \in V$
    \item $T(\alpha v) = \alpha T(v) \quad \forall v \in V \text{ and scalar }\alpha $
\end{itemize}

}
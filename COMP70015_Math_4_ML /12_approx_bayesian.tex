\chapter{Approximate Bayesian Inference for ML}
\section{Bayesian Inference in Non-Linear Models}

In the previous chapter, we observed that the mean of the posterior in linear regression is equivalent to the ridge regression estimate with regularisation. Specifically, we computed:

\[
\begin{aligned}
\boldsymbol{\mu} &= \left(\boldsymbol{X}^{\top}\boldsymbol{X} + \frac{\sigma^{2}}{\tau^{2}}\mathbf{I}\right)^{-1}\boldsymbol{X}^{\top}\boldsymbol{y}, \\
\boldsymbol{\Sigma} &= \left(\frac{1}{\sigma^{2}}\boldsymbol{X}^{\top}\boldsymbol{X} + \sigma^{2}\mathbf{I}\right)^{-1}.
\end{aligned}
\]

\begin{itemize}
    \item In the case of the normal distribution:
    \begin{itemize}
        \item The mean of the posterior corresponds to the parameter with the maximum posterior probability (MAP estimate).
        \item We have identified four estimators in this course: Ordinary Least Squares (OLS), Maximum Likelihood Estimation (MLE), Ridge Regression, and Maximum Posterior Probability (MAP).
    \end{itemize}
    \item Relationships between estimators:
    \begin{itemize}
        \item OLS = MLE
        \item Ridge = MAP
    \end{itemize}
    \item Importance:
    \begin{itemize}
        \item Viewing machine learning problems from different mathematical perspectives motivates sophisticated model choices.
    \end{itemize}
\end{itemize}

We can generalise our findings to compute the MAP estimate in logistic regression:
\begin{itemize}
    \item Recall the MAP estimate in linear regression derived from density matching:
    \[
    \log(p(\theta|\mathcal{D})) \propto \log(p(\mathcal{D}|\theta)) + \log(p(\theta)).
    \]
    \item \textbf{Choosing the Prior in Logistic Regression:}
    \begin{itemize}
        \item Parameters are not directly those of the Bernoulli distribution; they parameterise a linear transformation.
        \item No conjugacy exists in this model.
        \item A Gaussian prior acts as a regulariser by keeping parameter magnitudes small.
    \end{itemize}
    \item \textbf{Log Posterior Expression:}
    \[
    \log p(\theta|\mathcal{D}) \propto \sum_{i=1}^N \left( y^{(i)} \log \sigma(\mathbf{x}^{(i)\top}\theta) + (1 - y^{(i)}) \log \left(1 - \sigma(\mathbf{x}^{(i)\top}\theta)\right) \right) + \frac{1}{2\tau^2} \theta^\top \theta.
    \]
\end{itemize}

\textbf{Conclusion:}
\begin{itemize}
    \item There is no closed-form expression for the MAP estimate in logistic regression.
    \item We know this because there is no closed form MLE (close form of MAP when $\tau$ = 0)
    \item Instead, we must rely on using approximate Bayesian inference methods.
\end{itemize}


\section{Approximate Bayesian Inference}

Approximate Bayesian inference addresses the intractability of computing the posterior distribution by approximating it. There are two main categories:

\begin{itemize}
    \item \textbf{Sample-based Approximations}
    \item \textbf{Variational Approximations}
\end{itemize}

We will explore both categories, starting with approaches that provide insightful perspectives on Bayesian inference.

\subsection{Laplace Approximation}

The Laplace approximation approximates the posterior distribution with a Gaussian centred at the Maximum A Posteriori (MAP) estimate.

\begin{itemize}
    \item \textbf{Negative Log Posterior (Energy):}
    \[
    E(\theta) = -\log p(\theta, \mathcal{D}) = -\log p(\mathcal{D} \mid \theta) - \log p(\theta)
    \]
    \item \textbf{Properties:}
    \begin{itemize}
        \item Negative Log Posterior $E(\theta)$ is convex since both the negative log-likelihood and negative log-prior are convex. The sum of two convec functions is convex.
        \item Thus, gradient descent converges to the global minimum with many interations, which is the MAP estimate $\theta^{*}$.
    \end{itemize}
    \item \textbf{Second-Order Taylor Expansion around $\theta^{*}$:}
    \begin{itemize}
        \item To approximate the posterior, we first perform a second-order Taylor expansion of the energy $E(\theta)$ around its mode $\theta^{*}$, which is the MAP estimate: 
        \[
        \theta^* = \arg\min_{\theta} E(\theta).
        \]
        We expand and have:
    \end{itemize}
    \begin{align*}
        E(\theta) &\approx E(\theta^*) + (\theta - \theta^*)^\top \nabla E(\theta^*) + \frac{1}{2} (\theta - \theta^*)^\top \mathbf{H} (\theta - \theta^*) \\
        &\approx E(\theta^*) + \frac{1}{2} (\theta - \theta^*)^\top \mathbf{H} (\theta - \theta^*)
    \end{align*}
    Then, we approximate the posterior distribution as:
    $$p(\theta\mid\mathcal{D})\propto\exp(-E(\theta))\approx\exp\biggl(-E(\theta^{*})-\frac{1}{2}(\theta-\theta^{*})^{\top}\mathbf{H}(\theta-\theta^{*})\biggr).$$
    This expression resembles a multivariate Gaussian distribution, and thus we can write:
    \item \textbf{Gaussian Approximation of the Posterior:}
    \[
    \hat{p}(\theta \mid \mathcal{D}) \approx \mathcal{N}(\theta \mid \theta^*, \mathbf{H}^{-1})
    \]
    \begin{itemize}
        \item $\theta^{*}$: Mean (MAP estimate)
        \item $\mathbf{H}^{-1}$: Covariance matrix (inverse Hessian)
    \end{itemize}
    \item \textbf{Summary:}
    \begin{itemize}
        \item The Laplace approximation provides a Gaussian approximation centred at the MAP estimate.
        \item Efficient for models where the posterior is approximately Gaussian, such as Generalised Linear Models (GLMs).
    \end{itemize}
\end{itemize}

\subsection{Variational Inference}

Variational Inference (VI) approximates complex posterior distributions with simpler, tractable ones by solving an optimisation problem.

\begin{itemize}
    \item \textbf{Objective:}
    \begin{itemize}
        \item Select a variational family $q(\theta)$ that is easy to work with (e.g., Gaussian).
        \item Optimise the parameters of $q(\theta)$ to minimise the divergence from the true posterior $p(\theta \mid \mathcal{D})$.
    \end{itemize}
    \item \textbf{Advantages:}
    \begin{itemize}
        \item \textbf{Computational Efficiency:} Transforms Bayesian inference into an optimisation problem.
        \item \textbf{Scalability:} Suitable for large datasets and high-dimensional models.
    \end{itemize}
    \item \textbf{Common Variational Families:}
    \begin{itemize}
        \item \textbf{Mean-Field Approximation:}
        \[
        q(\theta) = \prod_i q(\theta_i)
        \]
        \item Assumes independence between parameters, simplifying optimisation.
    \end{itemize}
    \item \textbf{Trade-offs:}
    \begin{itemize}
        \item \textbf{Speed vs. Accuracy:} Faster than exact methods like MCMC but may be less accurate.
    \end{itemize}
\end{itemize}

\subsection{Markov Chain Monte Carlo Algorithms}

Markov Chain Monte Carlo (MCMC) algorithms generate samples from the true posterior distribution, enabling us to accurately estimate posterior properties. Unlike variational inference, which approximates the posterior directly with a simpler distribution, MCMC methods aim to generate samples from the true posterior distribution itself.

\begin{itemize}
    \item \textbf{Core Idea:}
    \begin{itemize}
        \item Construct a Markov chain with the desired posterior as its equilibrium distribution.
        \item Run the chain for sufficient steps to obtain representative samples.
    \end{itemize}
    \item \textbf{Common Algorithms:}
    \begin{itemize}
        \item \textbf{Metropolis-Hastings:}
        \begin{itemize}
            \item Iteratively proposes new samples.
            \item Accepts or rejects proposals based on posterior probabilities.
            \item Most widely used
        \end{itemize}
        \item \textbf{Gibbs Sampling:}
        \begin{itemize}
            \item Simplifies sampling by breaking down the process into conditional distributions for each parameter.
        \end{itemize}
    \end{itemize}
    \item \textbf{Advantages:}
    \begin{itemize}
        \item Highly accurate as they do not require simplifying assumptions about the posterior's shape.
        \item Applicable to multimodal or skewed posteriors.
    \end{itemize}
    \item \textbf{Drawbacks:}
    \begin{itemize}
        \item \textbf{Computational Expense:} Especially for high-dimensional models or complex dependencies.
    \end{itemize}
\end{itemize}

\section{Kinds of Predictive Uncertainty}

In Bayesian modelling, uncertainty is quantified to represent our beliefs about predictions. There are two primary types of predictive uncertainty:

\defb{Aleatoric Uncertainty}{
    Aleatoric uncertainty, also known as irreducible uncertainty, arises from the inherent noise in the observations. It cannot be reduced by gathering more data.
}

\defb{Epistemic Uncertainty}{
    Epistemic uncertainty, or reducible uncertainty, stems from our lack of knowledge about the underlying model parameters. It can be decreased by obtaining more data.
}

\subsection{Aleatoric Uncertainty}

Aleatoric uncertainty is present in supervised learning scenarios and can be categorised as:

\begin{itemize}
    \item \textbf{Homoscedastic Noise:}
    \begin{itemize}
        \item The observational noise is constant and does not vary with the input $\mathbf{x}$.
        \item Likelihood model:
        \[
        p(\mathcal{D} \mid \theta) = \prod_{i=1}^k \frac{1}{(2\pi\sigma^2)^{1/2}} \exp\left(-\frac{1}{2\sigma^2} (\mathbf{y}^{(i)} - \mathbf{x}^{(i)\top}\theta)^2 \right)
        \]
    \end{itemize}
    \item \textbf{Heteroscedastic Noise:}
    \begin{itemize}
        \item The observational noise varies with the input $\mathbf{x}$.
        \item Likelihood model:
        \[
        p(\mathcal{D} \mid \theta) = \prod_{i=1}^k \frac{1}{(2\pi\sigma(\mathbf{x})^2)^{1/2}} \exp\left(-\frac{1}{2\sigma(\mathbf{x})^2} (\mathbf{y}^{(i)} - \mathbf{x}^{(i)\top}\theta)^2 \right)
        \]
    \end{itemize}
    \item \textbf{Implications:}
    \begin{itemize}
        \item Aleatoric uncertainty is irreducible; adding more data to our training does not decrease this uncertainty.
        \item In neural networks, the variance can be treated as an additional output dimension to model heteroscedastic noise.
    \end{itemize}
\end{itemize}

\subsection{Epistemic Uncertainty}

Epistemic uncertainty arises from uncertainty in the model parameters and can be reduced with more data.

\begin{itemize}
    \item \textbf{Example:}
    \begin{itemize}
        \item Consider choosing an online seller:
        \begin{itemize}
            \item \textbf{Seller A:} 10 positive and 0 negative reviews.
                \[
                \text{Posterior Beta: } \text{Beta}(11, 1) \quad \text{Mean} = \frac{11}{12} \approx 0.9167, \quad \text{Variance} = 0.0059
                \]
            \item \textbf{Seller B:} 100 positive and 100 negative reviews.
                \[
                \text{Posterior Beta: } \text{Beta}(101, 101) \quad \text{Mean} = 0.5, \quad \text{Variance} = 0.0012
                \]
        \end{itemize}
    \end{itemize}
    \item \textbf{Characteristics:}
    \begin{itemize}
        \item Represents uncertainty about the correct model parameters.
        \item Can be reduced by acquiring more training data.
    \end{itemize}
    \item \textbf{Decomposition:}
    \begin{itemize}
        \item \textbf{Epistemic Uncertainty:} Uncertainty about model parameters.
        \item \textbf{Aleatoric Uncertainty:} Uncertainty from the data generation process.
    \end{itemize}
    \item \textbf{Application in Bayesian Active Learning:}
    \begin{itemize}
        \item Focuses on acquiring new training data that minimises epistemic uncertainty.
        \item Selects data points that provide the most information given current beliefs.
    \end{itemize}
\end{itemize}

\section{Bayesian Model Selection}

Bayesian model selection involves choosing the model that best explains the observed data by evaluating the model evidence or marginal likelihood.

\subsection{Model Evidence}

The model evidence, also known as the marginal likelihood, is defined for a model $m$ as:

\[
p(\mathcal{D} \mid m) = \int p(\mathcal{D} \mid \theta, m) \, p(\theta \mid m) \, d\theta
\]

\begin{itemize}
    \item Represents the probability of observing the data $\mathcal{D}$ under model $m$, integrating over all possible parameter values $\theta$.
    \item Balances:
    \begin{itemize}
        \item \textbf{Model Fit:} How well the model explains the data (likelihood).
        \item \textbf{Model Complexity:} The complexity of the model (prior).
    \end{itemize}
\end{itemize}

\subsection{Bayesian Model Selection Process}

\begin{itemize}
    \item Given a set of candidate models $\{m_1, m_2, \ldots, m_k\}$, compute the posterior probability for each model:
    \[
    p(m \mid \mathcal{D}) \propto p(\mathcal{D} \mid m) \, p(m)
    \]
    \begin{itemize}
        \item $p(m)$: Prior probability of model $m$.
        \item The model with the highest posterior probability is selected.
    \end{itemize}
    \item \textbf{Bayesian Occam's Razor:}
    \begin{itemize}
        \item Prefers simpler models unless the data strongly supports more complex ones.
        \item Complex models spread their probability mass over many possible datasets, leading to lower evidence for any specific dataset.
        \item Simpler models allocate more probability mass to the observed data, resulting in higher evidence.
    \end{itemize}
\end{itemize}

\subsection{Example: Logistic Regression Model Evidence}

Using the Laplace approximation, the model evidence for logistic regression can be approximated as:

\begin{itemize}
    \item \textbf{Laplace Approximation:}
    \[
    p(\theta \mid \mathcal{D}) \approx \mathcal{N}(\theta \mid \theta^*, \mathbf{H}^{-1})
    \]
    \item \textbf{Normalisation Constant (Marginal Likelihood):}
    \begin{align*}
        Z &= p(\mathcal{D}) \approx e^{-E(\theta^*)} (2\pi)^{\frac{D}{2}} |\mathbf{H}|^{-\frac{1}{2}} \\
        &= e^{-E(\theta^*)} (2\pi)^{\frac{D}{2}} \left|\mathbf{H}\right|^{-\frac{1}{2}}
    \end{align*}
    \begin{itemize}
        \item $D$: Dimensionality of $\theta$.
        \item $E(\theta^*)$: Negative log-posterior at the MAP estimate.
    \end{itemize}

    \begin{itemize}
        \item $e^{-E(\theta^*)}$ reflects the model fit to the data.
        \item $|\mathbf{H}|^{-\frac{1}{2}}$ penalises model complexity by accounting for the parameter space volume.
    \end{itemize}
\end{itemize}

\subsection{Summary}

\begin{itemize}
    \item Bayesian model selection utilises the model evidence to evaluate and compare different models.
    \item The approach inherently balances model fit and complexity, favouring simpler models unless justified by the data.
    \item Approximations like the Laplace method enable efficient computation of the model evidence, facilitating model comparison even in complex scenarios such as logistic regression.
\end{itemize}

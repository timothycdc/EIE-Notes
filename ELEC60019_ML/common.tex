%=======================PACKAGES FOR NEWCOMMAND CREATION========================
\usepackage{xparse}
%===============================================================================

%=========================PACKAGES FOR USE IN DOCUMENT==========================
\usepackage[dvipsnames]{xcolor}
\usepackage{graphicx, amssymb, amsfonts, amsmath, listings, multirow, hyperref, mathtools, svg, tikz, enumitem, minted, xfrac, multicol}
\usepackage[most]{tcolorbox}
\usepackage[super]{nth}
\usepackage{tabularx}  % for 'X' column type
\usepackage{multirow}  % for multirow cells
\usepackage{pgfplots}
\pgfplotsset{compat=1.17}
\usetikzlibrary{calc}

\usetikzlibrary{shapes.geometric, arrows.meta, positioning}

\tikzset{
	neuron/.style={circle, draw, minimum size=1cm},
	input neuron/.style={neuron, fill=gray!20},
	output neuron/.style={neuron, fill=gray!60},
	activation function/.style={
			rectangle,
			draw,
			minimum size=1cm,
			fill=gray!30,
			path picture={
					% Draw the step function inside the activation function node
					\draw[thick]
					($(path picture bounding box.south)+(-0.3cm,0.2cm)$) --
					($(path picture bounding box.south)+(0cm,0.2cm)$) --
					($(path picture bounding box.south)+(0cm,0.7cm)$) --
					($(path picture bounding box.south)+(0.3cm,0.7cm)$);
				}
		},
	arrow label/.style={midway, above}
}


% ======= CUSTOM 
\usepackage{subcaption}
\usepackage{wasysym}

%=================================GLOBAL SETUP==================================
\setitemize{itemsep=0em}
%===============================================================================

%===============================INCLUDE CHAPTERS================================
\newcommand{\addchapter}[1]{\include{#1/#1}}
%===============================================================================

%==============================UNFINISHED SECTION===============================
\newcommand{\unfinished}{\begin{huge} \textcolor{red}{\textbf{UNFINISHED!!!}} \end{huge}}
\newcommand{\toimprove}{\begin{huge} \textcolor{olive}{\textbf{NEEDS IMPROVEMENT!!!}} \end{huge}}
%===============================================================================

%==============================PAGE SPLIT LAYOUTS===============================
\NewDocumentCommand{\twosplit}{O{0.48} O{#1} m m}{
	\begin{minipage}[t]{#1\textwidth}
		#3
	\end{minipage}
	\hfill
	\begin{minipage}[t]{#2\textwidth}
		#4
	\end{minipage}
}
%===============================================================================

%============================SPECIAL COLOURED BOXES=============================

% For term definitions
% \begin{definitionbox}{term}
%	... the term's definition ...
% \end{definitionbox}
\newtcolorbox[auto counter,number within=section]{definitionbox}[2][]{%
	colback=blue!5!white,colframe=blue!75!black,arc=0mm,sharp corners=all,fonttitle=\bfseries,%
	title=#2 \hfill Definition \thetcbcounter #1}

% For term definitions
% \begin{sidenotebox}{cool title}
%	... cool unsassessed/not required info ...
% \end{sidenotebox}
\newtcolorbox[auto counter,number within=section]{sidenotebox}[2][]{%
	colback=black!5!white,colframe=black!75!black,arc=0mm,sharp corners=all,fonttitle=\bfseries,%
	title=#2 \hfill \textit{Extra, Not Assessed} \thetcbcounter #1}
% previously the text was "Extra Fun!"

% For example questions:
% \begin{examplebox}{question name}
%   ... the question ...
%	\tcblower
%   ... the worked answer ...
% \end{examplebox}
\newtcolorbox[auto counter,number within=section]{examplebox}[2][]{%
	colback=orange!5!white,breakable,colframe=orange!75!black,arc=0mm,sharp corners=all,fonttitle=\bfseries,%
	title=#2 \hfill Example Question \thetcbcounter #1}

% For exam questions (no answers):
% \begin{exambox}{1c}{2018}
%   ... the question ...
% \end{exambox}
\newtcolorbox[auto counter,number within=section]{exambox}[3][]{%
	colback=purple!5!white,breakable,colframe=purple!75!black,arc=0mm,sharp corners=all,fonttitle=\bfseries,
	title=Q#2 - #3 \hfill Exam Question \thetcbcounter #1}

% For comments:
% \begin{commentbox}
%   ... the question ...
% \end{commentbox}

\newtcolorbox[auto counter,number within=section]{commentbox}[2][]{%
	colback=orange!5!white,
	colframe=orange!95!black, % Darker orange frame
	coltitle=white, % White title text
	arc=0mm,
	sharp corners=all,
	fonttitle=\bfseries,
	title=#2 \hfill
}




% For positives/pros 
% \begin{prosbox}
%	... the term's definition ...
% \end{prosbox}
\newtcolorbox[]{prosbox}[1][]{%
	colback=green!5!white,breakable,colframe=green!75!black,leftrule=3mm,arc=0mm,sharp corners=all, #1}

% For negatives/cons 
% \begin{prosbox}
%	... the term's definition ...
% \end{prosbox}
\newtcolorbox[]{consbox}[1][]{%
	colback=red!5!white,breakable,colframe=red!75!black,leftrule=3mm,arc=0mm,sharp corners=all, #1}

% For negatives/cons 
% \begin{tabbox}{consbox}
%	... the term's definition ...
% \end{tabbox}
\newenvironment{tabbox}[2][.8\textwidth]{
	\def\boxtype{#2}
	\begin{\boxtype}
		\begin{center}
			\begin{tabular}{r p{#1}}
				}{
			\end{tabular}
		\end{center}
	\end{\boxtype}
}

% \begin{panoptobox}
%   ... the question ...
% \end{panoptobox}
\newtcolorbox{panoptobox}{enhanced,arc=0mm,breakable,sharp corners=all,colback=gray!5,colframe=gray,leftrule=12mm,detach title,%
	underlay unbroken and first={\node[below,text=black,anchor=east] at (interior.base west) {\includegraphics[width=11mm]{../common/images/panopto_logo.png}};}}

\newcommand{\lectlink}[2]{
	\begin{panoptobox}
		\textbf{\href{#1}{#2}}
	\end{panoptobox}
}
%===============================================================================